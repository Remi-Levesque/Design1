%!TEX encoding = IsoLatin

%
% Chapitre "Structure d'un rapport technique"
%

\chapter{Description}
\label{s:structure_rapport}

Dans l’optique d’améliorer la fiabilité des données de suivi des populations de poissons, le Ministère de la Faune Aquatique souhaite mesurer l’activité marine sur différents sites sauvages et commerciales. À l’aide du projet pilote Fish \& Chips, le Ministère souhaite trouver une solution qui comblerait l’ensemble de ses besoins. M. Bergevin a d’ailleurs été chargé par le Ministère pour trouver le design conceptuel le plus adapté et le plus efficace parmi les firmes d’ingénieurs. C’est pourquoi la firme d'ingénieur des Requins devra se pencher sur ce mandat et proposer une solution fiable qui respectera l'ensemble des besoins du client.

Afin de respecter les demandes du Ministère, il est nécessaire de concevoir un système autonome afin de dénombrer et de documenter la faune aquatique. Ce nouveau système se doit d’identifier et de comptabiliser différentes espèces de poissons à tout moment. L'ensemble des activités du système doivent également garantir une mesure passive, c'est-à-dire sans risque pour les poissons. Pour une durée de deux ans, le système se doit de compiler des données pour des raisons de validation et doit être facilement accessible par un opérateur. Les coûts et les délais nécessaires à la conception et la réalisation d'un tel système doivent être minimisés. Par ailleurs, l'importance de l'aspect esthétique du système est négligeable, dans la mesure où elle n'affecte pas la disponibilité du capteur.
