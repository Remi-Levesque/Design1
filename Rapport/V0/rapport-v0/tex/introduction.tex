%!TEX encoding = IsoLatin

%
% Chapitre "Introduction"
%

\chapter{Introduction}
\label{s:intro}

Au milieu des ann�es 1970, on assiste � l'une des plus grandes r�volutions commerciales de l'histoire : l'invention du premier ordinateur personnel. Dans les ann�es qui suivent, on voit appara�tre les balbutiements des multinationales informatiques d'aujourd'hui tels qu'Apple, Intel, Microsoft et bien d'autres. Vraisemblablement, une lutte � l'innovation s'est install�e. Seulement 40 ans plus tard, avoir sous la main un t�l�phone intelligent plus performant que l'ordinateur derri�re la mission Apollo 11, responsable de l'atterrissage du premier homme sur la lune, ne nous impressionne plus. Cependant, c'est gr�ce � cette constante �volution de la technologie qui nous entoure que sont venues de nombreuses opportunit�s d'affaires pour les firmes ing�nieurs.

Dans ce pr�sent projet, il sera justement question de d�velopper un design de capteur permettant la documentation de la faune aquatique dans un milieu donn�.

Le mandat fourni par le minist�re de la Faune Aquatique impose donc une identification pr�cise des populations de poissons, une collecte fiable d'images � des fins statistiques ainsi que l'acc�s � une base de donn�es. Bref, le d�veloppement de ce produit pourra se traduire en deux principaux aspects : l'implantation d'un logiciel capable de fournir des donn�es avec une fiabilit� et une s�curit� accrues, et la cr�ation d'un concept de capteur multidisciplinaire qui r�pond aux standards de qualit� du client.	




