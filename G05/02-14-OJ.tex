%!TEX encoding = IsoLatin

%
% Exemple d'ordre du jour
% par Pierre Tremblay, Universite Laval
% modifie par Christian Gagne, Universite Laval
% 2011/01/14 - version 1.4
% modifi� par Robert Bergevin, Universit� Laval
% 24/11/2011
% modifi� par Jean-Yves Chouinard, Universit� Laval
% 2016/01/11
% modifi� par Jean-Yves Chouinard, Universit� Laval
% 2017/01/04
%

%--------------------------------------------------------------------------------------
%------------------------------------- preambule --------------------------------------
%--------------------------------------------------------------------------------------
\documentclass[12pt]{ULojpv}

% Chargement des packages supplementaires
\usepackage[ansinew]{inputenc}

% Definitions des parametres de l'en-tete
\Cours{GEL--1001 Design I (m�thodologie)}             % Nom du cours
\NumeroEquipe{7}                                     % Numero de l'equipe
\NomEquipe{Les Requins}                               % Nom de l'equipe
\Objet{Ordre du jour}                                 % Nom du document
\SujetRencontre{Organisation et planification}        % Sujet de la rencontre
\DateRencontre{2017/01/16}                            % Date de la rencontre
\LocalRencontre{PLT--1122}                            % Local de la rencontre
\HeureRencontre{13h30-15h00}                          % Heure de la rencontre

%--------------------------------------------------------------------------------------
%--------------------------------- corps du document ----------------------------------
%--------------------------------------------------------------------------------------
\begin{document}
\entete
\begin{enumerate}
   \item \textbf{Ouverture de la r�union}
   \item \textbf{Nomination ou confirmation du pr�sident et du secr�taire}
   \item \textbf{Lecture et adoption de l'ordre du jour}
   \item \textbf{Lecture et adoption du proc�s-verbal de la r�union du 12 janvier 2017}
   \item \textbf{Affaires d�coulant du proc�s-verbal}
      \begin{enumerate}
         \item Investigations sur la cam�ra � utiliser
            \begin{enumerate}
               \item Consid�rations nocturnes
               \item Collecte de donn�es et compatibilit� avec le logiciel
               \item Co�ts
               \item Faisabilit� du logiciel
               \item Processus d'apprentissage et vision num�rique
            \end{enumerate}
      \end{enumerate}
   \item \textbf{Points � traiter}
      \begin{enumerate}
         \item Aspects m�caniques: mat�riaux de conception et durabilit�
            \begin{enumerate}
               \item Valider les contraintes maximales de la cam�ra ainsi que sa tenue � l'oxydation
               \item Dur�e de vie et temp�rature
            \end{enumerate}
         \item Aspects de s�curit�: cr�ation d'un r�seau de transfert de donn�es efficace
         \begin{enumerate}
               \item Types de transfert de donn�es possibles: �valuer les cas de perte de signal
               ou de vol.
               \item Adapter le signal � un milieu marin. (Quel type de signal serait le plus fiable sous l'eau et � distance?)
            \end{enumerate}
      \end{enumerate}
   \item \textbf{Divers}
   \item \textbf{R�partition des t�ches}
   \item \textbf{�valuation de la r�union}
   \item \textbf{Date, heure, lieu et objectif de la prochaine r�union}
   \item \textbf{Fermeture de la r�union}
\end{enumerate}

\end{document}
