%!TEX encoding = IsoLatin

%
% Exemple d'ordre du jour
% par Pierre Tremblay, Universite Laval
% modifie par Christian Gagne, Universite Laval
% 2011/01/14 - version 1.4
% modifi� par Robert Bergevin, Universit� Laval
% 24/11/2011
% modifi� par Jean-Yves Chouinard, Universit� Laval
% 2016/01/11
% modifi� par Jean-Yves Chouinard, Universit� Laval
% 2017/01/04
%

%--------------------------------------------------------------------------------------
%------------------------------------- preambule --------------------------------------
%--------------------------------------------------------------------------------------
\documentclass[12pt]{ULojpv}

% Chargement des packages supplementaires
\usepackage[ansinew]{inputenc}

% Definitions des parametres de l'en-tete
\Cours{GEL--1001 Design I (m�thodologie)}             % Nom du cours
\NumeroEquipe{7}                                     % Numero de l'equipe
\NomEquipe{}                               % Nom de l'equipe
\Objet{Ordre du jour}                                 % Nom du document
\SujetRencontre{Organisation et planification}        % Sujet de la rencontre
\DateRencontre{2019/01/24}                            % Date de la rencontre
\LocalRencontre{PLT--2708}                            % Local de la rencontre
\HeureRencontre{8h30-12h20}                          % Heure de la rencontre

%--------------------------------------------------------------------------------------
%--------------------------------- corps du document ----------------------------------
%--------------------------------------------------------------------------------------
\begin{document}
\entete
\begin{enumerate}
   \item \textbf{Ouverture de la r�union}
   \item \textbf{Nomination ou confirmation du pr�sident et du secr�taire}
   \item \textbf{Lecture et adoption de l'ordre du jour}
   \item \textbf{Lecture et adoption du proc�s-verbal de la r�union du 17 janvier 2019}
   \item \textbf{Affaires d�coulant du proc�s-verbal}
      \begin{enumerate}
         \item Introduction � l'�quipe de travail
            \begin{enumerate}
               \item Pr�sentation des membres de l'�quipe et de leur programme d'�tude
            \end{enumerate}
         \item Tutoriel Latex
	\item Br�ve pr�sentation d'id�es sur le capteur 
      \end{enumerate}
   \item \textbf{Points � traiter}
      \begin{enumerate}
         \item Id�es sur le capteur
            \begin{enumerate}
               \item Options utiles pour la capture des images 
               \item Mobilit� du capteur
	    \item Choix d'un design de capteur en fonction des diff�rentes consid�rations demand�es par le client
            \end{enumerate}
         \item Identification des poissons
	 \begin{enumerate}
		\item M�thode de s�paration des types de poissons
		\item Capacit� du syst�me � distinguer les cinq types
	  \end{enumerate}
      \end{enumerate}
   \item \textbf{Divers}
   \item \textbf{R�partition des t�ches}
   \item \textbf{�valuation de la r�union}
   \item \textbf{Date, heure, lieu et objectif de la prochaine r�union}
   \item \textbf{Fermeture de la r�union}
\end{enumerate}

\end{document}

