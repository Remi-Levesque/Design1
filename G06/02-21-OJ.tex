%!TEX encoding = IsoLatin

%
% Exemple d'ordre du jour
% par Pierre Tremblay, Universite Laval
% modifie par Christian Gagne, Universite Laval
% 2011/01/14 - version 1.4
% modifi\'e par Robert Bergevin, Universit\'e Laval
% 24/11/2011
% modifi\'e par Jean-Yves Chouinard, Universit\'e Laval
% 2016/01/11
\]
% moi\'e par Jean-Yves Chouinard, Universit\'e Laval
% 2017/01/04
%

%--------------------------------------------------------------------------------------
%------------------------------------- preambule --------------------------------------
%--------------------------------------------------------------------------------------
\documentclass[12pt]{ULojpv}

% Chargement des packages supplementaires
\usepackage[ansinew]{inputenc}

% Definitions des parametres de l'en-tete
\Cours{GEL--1001 Design I (m\'ethodologie)}             % Nom du cours
\NumeroEquipe{7}                                     % Numero de l'equipe
\NomEquipe{Les Requins}                               % Nom de l'equipe
\Objet{Ordre du jour \#6}                                 % Nom du document
\SujetRencontre{Capteurs et Rapport v1}        % Sujet de la rencontre
\DateRencontre{2019/02/21}                            % Date de la rencontre
\LocalRencontre{PLT--2708}                            % Local de la rencontre
\HeureRencontre{8h30-10h20}                          % Heure de la rencontre

%--------------------------------------------------------------------------------------
%--------------------------------- corps du document ----------------------------------
%--------------------------------------------------------------------------------------
\begin{document}
\entete
\begin{enumerate}
   \item \textbf{Ouverture de la r\'eunion}
   \item \textbf{Nomination ou confirmation du pr\'esident et du secr\'etaire}
   \item \textbf{Lecture et adoption de l'ordre du jour}
   \item \textbf{Lecture et adoption du proc\`es-verbal de la r\'eunion du 7 f\'evrier 2019}
   \item \textbf{Lecture et adoption de la version 1 du rapport}
   \item \textbf{Affaires d\'ecoulant du proc\`es-verbal}
      \begin{enumerate}
         \item Conception du capteur
            \begin{enumerate}
               \item \'Enum\'eration des capteurs possibles
               \item Utilit\'es des capteurs
               \item Co\^uts
               \item Consid\'erations physiques
            \end{enumerate}
      \end{enumerate}
   	\item Retour sur le rapport v1 
	\begin{enumerate}
		\item Chapitre 3 : Besoins et objectifs 
		\item Chapitre 4 : Cahier de charges
	\end{enumerate}
   \item \textbf{Divers}
   \item \textbf{R\'epartition des t\^aches}
   \item \textbf{\'Evaluation de la r\'eunion}
   \item \textbf{Date, heure, lieu et objectif de la prochaine r\'eunion}
   \item \textbf{Fermeture de la r\'eunion}
\end{enumerate}

\end{document}
