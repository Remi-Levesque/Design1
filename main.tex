%!TEX encoding = IsoLatin

%
% Exemple de rapport
% par Pierre Tremblay, Universite Laval
% modifié par Christian Gagne, Universite Laval
% 14/01/2011 - version 1.3
% modifié par Robert Bergevin, Université Laval
% 24/11/2011
% modifié par Jean-Yves Chouinard, Université Laval
% 11/01/2016
% modifié par Jean-Yves Chouinard, Université Laval
% 04/01/2017
%

%
% Modele d'organisation d'un projet LaTeX 
% rapport/      dossier racine et fichier principal
% rapport/fig   fichiers des figures
% rapport/tex   autres fichiers .tex
%

% ** Preambule **
%
% Ajouter les options au besoin :
%    - "ULlof" pour inclure la liste des figures, requis si "\begin{figure}" utilise
%    - "ULlot" pour inclure la liste des tableaux, requis si "\begin{table}" utilise
%
\documentclass[12pt,ULlof,ULlot]{ULrapport}

% Chargement des packages supplementaires (si absent de la classe)
\usepackage[utf8x]{inputenc}
\usepackage[autolanguage]{numprint}
\usepackage{icomma}
%\usepackage[options]{nom_du_package}

% Definition d'une commande pour presenter des cellules multilignes dans un tableau
\newcommand{\cellulemultiligne}[1]{\begin{tabular}{@{}c@{}}#1\end{tabular}}

% Definition de colonnes en mode paragraphe avec alignement ajustable
% Cette definition requiert le chargement du package "array"
%    - alignement horizontal, parametre #1 : - \raggedright (aligne a gauche)
%                                            - \centering (centre)
%                                            - \raggedleft (aligne a droite)
%    - alignement vertical, parametre #2 : - p (aligne en haut)
%                                          - m (centre)
%                                          - b (aligne en bas)
%    - largeur, parametre #3 : longueur
\newcolumntype{Z}[3]{>{#1\hspace{0pt}\arraybackslash}#2{#3}}

% Definitions des parametres de la page titre
\TitreProjet{Fish \& Chips}                         % Titre du projet
\TitreRapport{Rapport de projet -- version 0}                       % Titre du rapport
\Destinataire{Robert Bergevin, Luc Lamontagne et Simon Thibault}         % Nom(s) du destinataire
\NumeroEquipe{7}                                     % Numero de l'equipe
\NomEquipe{Les Requins}                               % Nom de l'equipe
\TableauMembres{%                                     % Tableau des membres de l'equipe
   111\,239\,483  & Vincent Lambert    & \\\hline        % matricule & nom & \\\hline
   111\,238\,936  & Rémi Lévesque & \\\hline        % matricule & nom & \\\hline
   111\,171\,798  & Ibrahim Mahamadou & \\\hline        % matricule & nom & \\\hline
   111\,233\,742  & Honoré Marcotte & \\\hline        % matricule & nom & \\\hline
   111\,160\,242  & Jérémy Talbot-Pâquet& \\\hline        % matricule & nom & \\\hline
}
\DateRemise{31 janvier 2019}                           % Date de remis


% Contenu de l'historique des versions
\HistoriqueVersions{%                        % Version & Date & Description \\\hline
         & 30 janvier 2019 & Création du document \\\hline
   0   & 31 janvier 2019 & Mise en page, ajout de la table des matières, des chapitres d'introduction et de description du projet\\\hline
   1   & 21 févier 2019 & Ajout du chapitre «Objectifs» et rédaction du cahier des charges\\\hline
   2   & 21 mars 2019 & Ajout du chapitre «Conceptualisation et analyse de faisabilité»\\\hline
   Finale   & 18 avril 2019 & Ajout des chapitres «Étude préliminaire» et «Concept retenu»\\\hline
}


% Corps du document

\begin{document}

%   Chapitres
%!TEX encoding = IsoLatin

%
% Chapitre "Introduction"
%

\chapter{Introduction}
\label{s:intro}

Au milieu des années 1970, on assiste à l'une des plus grandes révolutions commerciales de l'histoire : l'invention du premier ordinateur personnel. Dans les années qui suivent, on voit apparaître les balbutiements des multinationales informatiques d'aujourd'hui tels qu'Apple, Intel, Microsoft et bien d'autres. Vraisemblablement, une lutte à l'innovation s'est installée. Seulement 40 ans plus tard, avoir sous la main un téléphone intelligent plus performant que l'ordinateur derrière la mission Apollo 11, responsable de l'atterrissage du premier homme sur la lune, ne nous impressionne plus. Cependant, c'est grâce à cette constante évolution de la technologie qui nous entoure que sont venues de nombreuses opportunités d'affaires pour les firmes ingénieurs.

Dans ce présent projet, il sera justement question de développer un design de capteur permettant la documentation de la faune aquatique dans un milieu donné.

Le mandat fourni par le ministère de la Faune Aquatique impose donc une identification précise des populations de poissons, une collecte fiable d'images à des fins statistiques ainsi que l'accès à une base de données. Bref, le développement de ce produit pourra se traduire en deux principaux aspects : l'implantation d'un logiciel capable de fournir des données avec une fiabilité et une sécurité accrues, et la création d'un concept de capteur multidisciplinaire qui répond aux standards de qualité du client.	





%!TEX encoding = IsoLatin

%
% Chapitre "Structure d'un rapport technique"
%

\chapter{Description du projet}
\label{s:structure_rapport}

De nos jours, l'accès à l'information est plus important que jamais. C'est dans ce contexte que le Ministère de la Faune Aquatique fait un pas en avant à l'aide du projet pilote Fish and Chips. En effet, le Ministère de la Faune Aquatique se soucie des données statistiques provenant des populations de poissons. Plus précisément, il souhaite mesurer l'activité marine sur différents sites afin d'améliorer la fiabilité des données de suivis des mammifères marins. Le Ministère de la Faune Aquatique désire également compiler ces données de manière confidentielle à des fins statistiques. La firme de génie-conseil des Requins devra donc se pencher sur ce mandat et proposer une solution fiable qui comblera l'ensemble des besoins du client.

Afin de respecter les demandes du Ministère, il est nécessaire de concevoir un système autonome et fixe afin de dénombrer et de documenter la faune aquatique. Ce nouveau système se doit d'être automatisé et complètement autonome pour comptabiliser et identifier différentes espèces de poissons à tout moment. En ce sens, la qualité des informations et des mesures prises est primordiale. De plus, l'appareillage doit être muni d'un capteur afin de recueillir des images et de détailler diverses statistiques sur le territoire. L'ensemble des activités du système doivent également garantir une mesure passive, c'est-à-dire sans risque pour les poissons. Le Ministère de la Faune Aquatique souhaite que la communication avec le système concernant la configuration et les opérations s'effectue à distance sous une connexion sécurisée, et ce, dans l'optique de minimiser le contact humain avec les espèces aquatiques. Par ailleurs, les frais des relevés de terrain s'en trouveront aussi diminués. Pour une durée de deux ans, le système se doit de compiler des données pour des raisons de validation et doit être en mesure d'acheminer une alarme à un opérateur en cas de défectuosité. Les coûts et les délais nécessaires à la conception et la réalisation d'un tel système doivent être minimisés. De plus, une reconfiguration de l'appareil doit être possible afin qu'il soit adapté au site où il sera implanté. Par ailleurs, l'importance de l'aspect esthétique du système est négligeable, dans la mesure où elle n'affecte pas la disponibilité du capteur.

%!TEX encoding = IsoLatin

%
% Chapitre "Objectifs"
%

\chapter{Besoins et objectifs}
\label{s:objectifs}

\section{Analyse des besoins}

Afin de bien saisir la demande du client et de lui fournir une solution appropriée, une analyse des besoins sera réalisée. Pour commencer, l'automatisation et l'autonomie seront au coeur de ce projet. Notamment, la prise de photo devra être faite automatiquement. Le système se devra donc d'avoir un dispositif lui permettant de savoir quand prendre des photos et savoir si la photo contient bel et bien un poisson. L'enregistrement des données doit aussi se faire automatiquement. Après la collecte de données, le système sera tenu de stocker les données par lui-même. Ensuite, il faudra gérer l'identification des poissons. Pour que le système soit efficace, il devra être en mesure d'identifier jusqu'à cinq variétés de poissons différentes, et ce, sans intervention humaine. Dans la même lancée, le système devra être autonome pour effectuer ces fonctions. Il devra pouvoir fonctionner pendant au moins 2 semaines avant d'avoir recours à une maintenance.

Un autre besoin important consiste en la possibilité de communiquer à distance avec le système. Plus concrètement, l'utilisateur devra être capable d'avoir accès aux données peu importe sa localisation et en tout temps. Puisque cela implique nécessairement la conception d'un serveur. Il faudra donc que le système de prise de photo puisse communiquer avec ce serveur via une connexion locale. De plus, cela donnera la possibilité à l'utilisateur de pouvoir contrôler à distance certains paramètres sur l'opération du capteur.

Par souci de confidentialité des renseignements et des données, toutes les connexions devront être sécurisées. Seul un utilisateur ayant une autorisation pourra communiquer avec le système.

Ensuite, les données enregistrées telles que les images prises par le capteur et les informations sur le système devront être stockées et accessibles pour une durée de 2 ans.

Finalement, le système devra assurer une qualité d'image assez bonne pour permettre la reconnaissance du poisson, et ce, même la nuit.

\section{Objectifs}

\begin{enumerate}
    \item Minimiser l'intervention humaine
    \begin{itemize}
        \item Maximiser la durée de vie de la batterie
        \item Maximiser la précision de l'identification des poissons
        \item Minimiser la complexité de la maintenance
        \item Automatiser le transfert de données
        \item Automatiser la prise de photos
    \end{itemize}
    
    \item Assurer la rentabilité du produit lors de la revente
    \begin{itemize}
        \item Minimiser le temps de conception
        \item Minimiser la complexité de l'usinage des pièces
        \item Faciliter la rechange des pièces
    \end{itemize}
    
    \item Minimiser les coûts
    \begin{itemize}
        \item Minimiser le temps d'apprentissage du logiciel de reconnaissance
        \item Minimiser les frais d'installation
        \item Minimiser les frais de maintenance
        \item Minimiser les frais d'opération
        \item Minimiser le coût des pièces
    \end{itemize}
    
    \item Assurer un produit de qualité
    \begin{itemize}
        \item Maximiser la durée de vie totale du système
        \item Assurer la modernité du logiciel de reconnaissance
        \item Donner une interface intuitive et au goût du jour
    \end{itemize}
    
    \item Maximiser la capacité de stockage des données
    \begin{itemize}
        \item Jsuis tanné
    \end{itemize}
    
    \item Autres (je sais pas dans quelle catégorie les mettre)
    \begin{itemize}
        \item Maximiser la facilité de conception
        \item Faciliter la distribution du capteur à plusieurs endroits
        \item Maximiser la disponibilité du capteur
        \item Maximiser la sécurité
        \item Assurer une mesure passive
        \item Assurer une qualité constante (précision)
        \item Maximiser les variétés de poissons identifiables
        \item Tout ça fuck un peu la structure
    \end{itemize}
    
\end{enumerate}

%%!TEX encoding = IsoLatin

%
% Chapitre "Cahier des charges"
%

\chapter{Cahier des charges}
\label{s:cahier_des_charges}


%%!TEX encoding = IsoLatin

%
% Chapitre "Conceptualisation et analyse de faisabilité"
%

\chapter{Conceptualisation et analyse de faisabilité}
\label{s:conceptualisation_et_analyse}


%%!TEX encoding = IsoLatin

%
% Chapitre "�tude pr�liminaire"
%

\chapter{Étude préliminaire}
\label{s:etude_preliminaire}


%%!TEX encoding = IsoLatin

%
% Chapitre "Concept retenu"
%

\chapter{Concept retenu}
\label{s:concept_retenu}


%%!TEX encoding = IsoLatin

%
% Chapitre "Bibliographie"
%

%\begin{thebibliographyUL}{99} % remplacer le "{9}" par "{99}" lorsque le nombre de references
                              % requiert 2 caracteres (>= 10 references)

%\end{thebibliographyUL}

%   Annexes
%\appendix
%%!TEX encoding = IsoLatin

%
% Annexe "Liste des sigles et des acronymes"
%

\chapter{Liste des sigles et des acronymes}

% Ne pas y inclure les unités SI

\begin{flushleft}
   \begin{tabular}{@{}ll}
      BIPM        & Bureau international des poids et mesures        \\
      CGPM        & Conférence générale des poids et mesures         \\
      CODATA      & Committee on Data for Science and Technology     \\
      ISBN        & International Standard Book Number               \\
      JPEG        & Joint Photographic Experts Group                 \\
      NIST        & National Institute of Standards and Technology   \\
      PDF         & Portable Document Format                         \\
      RADARSAT    & RADAR SATellite                                  \\
      SI          & Système international d'unités                   \\
      URL         & Uniform Resource Locator                         \\
   \end{tabular}
\end{flushleft}








\end{document}
% Fin du document

