%!TEX encoding = IsoLatin

%
% Chapitre "Structure d'un rapport technique"
%

\chapter{Description du projet}
\label{s:structure_rapport}

De nos jours, l'accès à l'information est plus important que jamais. C'est dans ce contexte que le Ministère de la Faune Aquatique fait un pas en avant à l'aide du projet pilote Fish and Chips. En effet, le Ministère de la Faune Aquatique se soucie des données statistiques provenant des populations de poissons. Plus précisément, il souhaite mesurer l'activité marine sur différents sites afin d'améliorer la fiabilité des données de suivis des mammifères marins. Le Ministère de la Faune Aquatique désire également compiler ces données de manière confidentielle à des fins statistiques. La firme de génie-conseil des Requins devra donc se pencher sur ce mandat et proposer une solution fiable qui comblera l'ensemble des besoins du client.

Afin de respecter les demandes du Ministère, il est nécessaire de concevoir un système autonome et fixe afin de dénombrer et de documenter la faune aquatique. Ce nouveau système se doit d'être automatisé et complètement autonome pour comptabiliser et identifier différentes espèces de poissons à tout moment. En ce sens, la qualité des informations et des mesures prises est primordiale. De plus, l'appareillage doit être muni d'un capteur afin de recueillir des images et de détailler diverses statistiques sur le territoire. L'ensemble des activités du système doivent également garantir une mesure passive, c'est-à-dire sans risque pour les poissons. Le Ministère de la Faune Aquatique souhaite que la communication avec le système concernant la configuration et les opérations s'effectue à distance sous une connexion sécurisée, et ce, dans l'optique de minimiser le contact humain avec les espèces aquatiques. Par ailleurs, les frais des relevés de terrain s'en trouveront aussi diminués. Pour une durée de deux ans, le système se doit de compiler des données pour des raisons de validation et doit être en mesure d'acheminer une alarme à un opérateur en cas de défectuosité. Les coûts et les délais nécessaires à la conception et la réalisation d'un tel système doivent être minimisés. De plus, une reconfiguration de l'appareil doit être possible afin qu'il soit adapté au site où il sera implanté. Par ailleurs, l'importance de l'aspect esthétique du système est négligeable, dans la mesure où elle n'affecte pas la disponibilité du capteur.
