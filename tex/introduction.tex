%!TEX encoding = IsoLatin

%
% Chapitre "Introduction"
%

\chapter{Introduction}
\label{s:intro}

Au milieu des années 1970, on assiste à l'une des plus grandes révolutions commerciales de l'histoire : l'invention du premier ordinateur personnel. Dans les années qui suivent, on voit apparaître les balbutiements des multinationales informatiques d'aujourd'hui tels qu'Apple, Intel, Microsoft et bien d'autres. Vraisemblablement, une lutte à l'innovation s'est installée. Seulement 40 ans plus tard, avoir sous la main un téléphone intelligent plus performant que l'ordinateur derrière la mission Apollo 11, responsable de l'atterrissage du premier homme sur la lune, ne nous impressionne plus. Cependant, c'est grâce à cette constante évolution de la technologie qui nous entoure que sont venues de nombreuses opportunités d'affaires pour les firmes ingénieurs.

Dans ce présent projet, il sera justement question de développer un design de capteur permettant la documentation de la faune aquatique dans un milieu donné.

Le mandat fourni par le ministère de la Faune Aquatique impose donc une identification précise des populations de poissons, une collecte fiable d'images à des fins statistiques ainsi que l'accès à une base de données. Bref, le développement de ce produit pourra se traduire en deux principaux aspects : l'implantation d'un logiciel capable de fournir des données avec une fiabilité et une sécurité accrues, et la création d'un concept de capteur multidisciplinaire qui répond aux standards de qualité du client.	




