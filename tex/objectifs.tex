%!TEX encoding = IsoLatin

%
% Chapitre "Objectifs"
%

\chapter{Besoins et objectifs}
\label{s:objectifs}

\section{Analyse des besoins}

Afin de bien saisir la demande du client et de lui fournir une solution appropriée, une analyse des besoins sera réalisée. Pour commencer, l'automatisation et l'autonomie seront au coeur de ce projet. Notamment, la prise de photo devra être faite automatiquement. Le système se devra donc d'avoir un dispositif lui permettant de savoir quand prendre des photos et savoir si la photo contient bel et bien un poisson. L'enregistrement des données doit aussi se faire automatiquement. Après la collecte de données, le système sera tenu de stocker les données par lui-même. Ensuite, il faudra gérer l'identification des poissons. Pour que le système soit efficace, il devra être en mesure d'identifier jusqu'à cinq variétés de poissons différentes, et ce, sans intervention humaine. Dans la même lancée, le système devra être autonome pour effectuer ces fonctions. Il devra pouvoir fonctionner pendant au moins 2 semaines avant d'avoir recours à une maintenance.

Un autre besoin important consiste en la possibilité de communiquer à distance avec le système. Plus concrètement, l'utilisateur devra être capable d'avoir accès aux données peu importe sa localisation et en tout temps. Puisque cela implique nécessairement la conception d'un serveur. Il faudra donc que le système de prise de photo puisse communiquer avec ce serveur via une connexion locale. De plus, cela donnera la possibilité à l'utilisateur de pouvoir contrôler à distance certains paramètres sur l'opération du capteur.

Par souci de confidentialité des renseignements et des données, toutes les connexions devront être sécurisées. Seul un utilisateur ayant une autorisation pourra communiquer avec le système.

Ensuite, les données enregistrées telles que les images prises par le capteur et les informations sur le système devront être stockées et accessibles pour une durée de 2 ans.

Finalement, le système devra assurer une qualité d'image assez bonne pour permettre la reconnaissance du poisson, et ce, même la nuit.

\section{Objectifs}

\begin{enumerate}
    \item Minimiser l'intervention humaine
    \begin{itemize}
        \item Maximiser la durée de vie de la batterie
        \item Maximiser la précision de l'identification des poissons
        \item Minimiser la complexité de la maintenance
        \item Automatiser le transfert de données
        \item Automatiser la prise de photos
    \end{itemize}
    
    \item Assurer la rentabilité du produit lors de la revente
    \begin{itemize}
        \item Minimiser le temps de conception
        \item Minimiser la complexité de l'usinage des pièces
        \item Faciliter la rechange des pièces
    \end{itemize}
    
    \item Minimiser les coûts
    \begin{itemize}
        \item Minimiser le temps d'apprentissage du logiciel de reconnaissance
        \item Minimiser les frais d'installation
        \item Minimiser les frais de maintenance
        \item Minimiser les frais d'opération
        \item Minimiser le coût des pièces
    \end{itemize}
    
    \item Assurer un produit de qualité
    \begin{itemize}
        \item Maximiser la durée de vie totale du système
        \item Assurer la modernité du logiciel de reconnaissance
        \item Donner une interface intuitive et au goût du jour
    \end{itemize}
    
    \item Maximiser la capacité de stockage des données
    \begin{itemize}
        \item Jsuis tanné
    \end{itemize}
    
    \item Autres (je sais pas dans quelle catégorie les mettre)
    \begin{itemize}
        \item Maximiser la facilité de conception
        \item Faciliter la distribution du capteur à plusieurs endroits
        \item Maximiser la disponibilité du capteur
        \item Maximiser la sécurité
        \item Assurer une mesure passive
        \item Assurer une qualité constante (précision)
        \item Maximiser les variétés de poissons identifiables
        \item Tout ça fuck un peu la structure
    \end{itemize}
    
\end{enumerate}
