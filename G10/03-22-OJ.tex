%!TEX encoding = IsoLatin

%
% Exemple d'ordre du jour
% par Pierre Tremblay, Universite Laval
% modifie par Christian Gagne, Universite Laval
% 2011/01/14 - version 1.4
% modifi� par Robert Bergevin, Universit� Laval
% 24/11/2011
% modifi� par Jean-Yves Chouinard, Universit� Laval
% 2016/01/11
% modifi� par Jean-Yves Chouinard, Universit� Laval
% 2017/01/04
% refonte compl�te par Dominique Beaulieu
% 2018-04-04
%

%--------------------------------------------------------------------------------------
%------------------------------------- pr�ambule --------------------------------------
%--------------------------------------------------------------------------------------
\documentclass[12pt]{ULojpv}

% Chargement des packages supplementaires
\usepackage[ansinew]{inputenc}
\usepackage[autolanguage]{numprint}

% Definitions des parametres de l'en-tete
\Cours{GEL--1001 Design I (m�thodologie)}             % Nom du cours
\NumeroEquipe{7}                                     % Numero de l'equipe
\NomEquipe{Les Requins}                               % Nom de l'equipe
\Objet{Ordre du jour 10}                                 % Nom du document
\SujetRencontre{Retour sur la conceptualisation et l'analyse de faisabilit�}        % Sujet de la rencontre
\DateRencontre{2019/03/21}                            % Date de la rencontre aaaa/mm/jj
\LocalRencontre{PLT--2708}                            % Local de la rencontre
\HeureRencontre{8h30-10h20}                          % Heure de la rencontre

%--------------------------------------------------------------------------------------
%--------------------------------- corps du document ----------------------------------
%--------------------------------------------------------------------------------------
\begin{document}
\entete
\begin{enumerate}
   \item \textbf{Ouverture de la r�union}
   \item \textbf{Nomination ou confirmation du pr�sident et du secr�taire}
   \item \textbf{Lecture et adoption de l'ordre du jour}
   \item \textbf{Lecture et adoption du proc�s-verbal de la r�union du 14 mars 2019}
   \item \textbf{Affaires d�coulant du proc�s-verbal}
      \begin{enumerate}
         \item Correction du rapport V1
            \begin{enumerate}
               \item Les crit�res d'�valuation
               \item Les bar�mes du cahier des charges
            \end{enumerate}
         \item Retour sur le travail effectu� en vue du rapport V2
		\begin{enumerate}
			\item Intrants, fonctions et extrants
			\item Correction s'il y a lieu du diagramme fonctionnel
			\item Analyse de faisabilit�
		\end{enumerate}
      \end{enumerate}
   \item \textbf{Points � traiter}
      \begin{enumerate}
         \item Elaboration du tableau de synth�se
         \item Remise du rapport V2
		
      \end{enumerate}
   \item \textbf{Divers}
   \item \textbf{R�partition des t�ches}
   \item \textbf{�valuation de la r�union}
   \item \textbf{Date, heure, lieu et objectif de la prochaine r�union}
   \item \textbf{Fermeture de la r�union}
\end{enumerate}

\end{document}

