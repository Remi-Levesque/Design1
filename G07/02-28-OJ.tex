%!TEX encoding = IsoLatin

%
% Exemple d'ordre du jour
% par Pierre Tremblay, Universite Laval
% modifie par Christian Gagne, Universite Laval
% 2011/01/14 - version 1.4
% modifi� par Robert Bergevin, Universit� Laval
% 24/11/2011
% modifi� par Jean-Yves Chouinard, Universit� Laval
% 2016/01/11
% modifi� par Jean-Yves Chouinard, Universit� Laval
% 2017/01/04
%

%--------------------------------------------------------------------------------------
%------------------------------------- preambule --------------------------------------
%--------------------------------------------------------------------------------------
\documentclass[12pt]{ULojpv}

% Chargement des packages supplementaires
\usepackage[ansinew]{inputenc}

% Definitions des parametres de l'en-tete
\Cours{GEL--1001 Design I (m�thodologie)}             % Nom du cours
\NumeroEquipe{7}                                     % Numero de l'equipe
\NomEquipe{Les Requins}                               % Nom de l'equipe
\Objet{Ordre du jour \#7}                                 % Nom du document
\SujetRencontre{Retour des concepts de solutions}        % Sujet de la rencontre
\DateRencontre{2019/02/28}                            % Date de la rencontre
\LocalRencontre{PLT--2708}                            % Local de la rencontre
\HeureRencontre{8h30-10h20}                          % Heure de la rencontre

%--------------------------------------------------------------------------------------
%--------------------------------- corps du document ----------------------------------
%--------------------------------------------------------------------------------------
\begin{document}
\entete
\begin{enumerate}
   \item \textbf{Ouverture de la r�union}
   \item \textbf{Nomination ou confirmation du pr�sident et du secr�taire}
   \item \textbf{Lecture et adoption de l'ordre du jour}
   \item \textbf{Lecture et adoption du proc�s-verbal de la r�union du 21 f�vrier 2019}
   \item \textbf{Affaires d�coulant du proc�s-verbal}

   \item \textbf{Points � traiter}
      \begin{enumerate}
	         \item �num�ration des concepts de solution retenus
	\begin{enumerate}
		\item Capteurs retenus
		\item Choix des batteries
		\item Syst�me de reconnaissance des poissons
		\item Capacit� et syst�me de stockage
		\item Acc�s au syst�me
		\item Syst�me de s�curit�
	\end{enumerate}
		\item �valuation du concept de solution \#1 selon les contraintes et les objectifs
	\begin{enumerate}
		\item Intervention humaine
		\begin{enumerate}
			\item Dur�e de vie de la batterie
			\item Complexit� de la maintenance
			\item Automatisation du transfert de donn�es
			\item Acc�s � distance
		\end{enumerate}
		\item Qualit� du produit
		\begin{enumerate}
			\item Dur�e de vie de l'appareil
			\item Pr�cision du logiciel de reconnaissance
			\item Utilisation de l'interface graphique
			\item Identification des poissons
			\item Capacit� de stockage des donn�es
			\item Fiabilit� du syst�me de s�curit�
		 \end{enumerate}
	\end{enumerate}
      \end{enumerate}
   \item \textbf{Divers}
   \item \textbf{R�partition des t�ches}
   \item \textbf{�valuation de la r�union}
   \item \textbf{Date, heure, lieu et objectif de la prochaine r�union}
   \item \textbf{Fermeture de la r�union}
\end{enumerate}

\end{document}
