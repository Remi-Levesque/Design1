%!TEX encoding = IsoLatin

%
% Exemple d'ordre du jour
% par Pierre Tremblay, Universite Laval
% modifie par Christian Gagne, Universite Laval
% 2011/01/14 - version 1.4
% modifi� par Robert Bergevin, Universit� Laval
% 24/11/2011
% modifi� par Jean-Yves Chouinard, Universit� Laval
% 2016/01/11
% modifi� par Jean-Yves Chouinard, Universit� Laval
% 2017/01/04
%

%--------------------------------------------------------------------------------------
%------------------------------------- preambule --------------------------------------
%--------------------------------------------------------------------------------------
\documentclass[12pt]{ULojpv}

% Chargement des packages supplementaires
\usepackage[ansinew]{inputenc}

% Definitions des parametres de l'en-tete
\Cours{GEL--1001 Design I (m�thodologie)}             % Nom du cours
\NumeroEquipe{7}                                     % Numero de l'equipe
\NomEquipe{Les requins}                               % Nom de l'equipe
\Objet{Ordre du jour \#4}                                 % Nom du document
\SujetRencontre{Cahier des charges et �quipements}        % Sujet de la rencontre
\DateRencontre{2019/02/07}                            % Date de la rencontre
\LocalRencontre{PLT--2708}                            % Local de la rencontre
\HeureRencontre{8h30-10h20}                          % Heure de la rencontre

%--------------------------------------------------------------------------------------
%--------------------------------- corps du document ----------------------------------
%--------------------------------------------------------------------------------------
\begin{document}
\entete
\begin{enumerate}

	\item \textbf{Ouverture de la r�union}

	\item \textbf{Nomination ou confirmation du pr�sident et du secr�taire}

	\item \textbf{Lecture et adoption de l'ordre du jour de la rencontre du 5 f�vrier 2019}

	\item \textbf{Lecture et adoption du proc�s-verbal de la r�union du 31 janvier 2019}

	\item \textbf{Affaires d�coulant du proc�s-verbal}

	\begin{enumerate}
		
		\item Retour sur le logiciel de reconnaissance par intelligence artificielle

		\item Retour sur le prix et les sp�cifications du capteur

		\item Retour sur le focus de la cam�ra et la zone de prise d'images

		\item Retour sur la mesure de la temp�rature

		\item Mise � jour sur l'avancement du cahier des charges
		
	\end{enumerate}

	\item \textbf{Points � traiter}
	
	\begin{enumerate}
	
		\item Id�es pour le logiciel de reconnaissance par intelligence artificielle
	
		\begin{enumerate}

			\item Possibilit� de la r�alisation
		
			\item Co�t du logiciel

		\end{enumerate}

		\item Id�es sur la prise d'images

		\begin{enumerate}

			\item Vision de nuit et flash infrarouge

		\end{enumerate}

		\item Cahier des charges

		\begin{enumerate}

			\item Exemple du cahier des charges

		\end{enumerate}

	\end{enumerate}

	\item \textbf{Divers}

	\item \textbf{R�partition des t�ches}

	\item \textbf{�valuation de la r�union}

	\item \textbf{Date, heure, lieu et objectif de la prochaine r�union}

	\item \textbf{Fermeture de la r�union}

\end{enumerate}

\end{document}

